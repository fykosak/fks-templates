\expandafter\ifx\csname classoptions\endcsname\relax
  \def\classoptions{}
\fi
\documentclass[10pt,fykos,\classoptions]{fksempty}

\usepackage{xstring}
\usepackage{fp}
\usepackage{pst-barcode}

\geometry{ignoreheadfoot,
  a4paper,
  hmargin=0.7cm,
  tmargin=0.20cm,
  bmargin=0.0cm,
  headsep=0.0cm}

\usepackage[fyziklani2]{fkslegacyloader}

\problemsdir{./problems}

% this class doesn't support those, but we use same common
\newcommand{\leftheader}[1]{\relax}
\newcommand{\rightheader}[1]{\relax}
\problemsdir{./problems}
\renewcommand\problemnum[2]{\arabic{#2}}
\renewcommand\problemstats{\relax}
\renewcommand\problempoints{5 – 3 – 2 – 1}
\leftheader{MFNáboj}
\rightheader{I. ročník \qquad 13. února 2012}

\input{naboj_team.tex}

% needed to use before \barcode or \qrcode
% produces 9 digit code according to https://en.wikipedia.org/wiki/Routing_transit_number#Check_digit
\newcommand{\barcodedef}[1]{%
    \def\probnumfull{\ifnum\theproblem<10 0\fi\theproblem}%
    \def\teamnumfull{\ifnum#1<100000 0\fi\ifnum#1<10000 0\fi\ifnum#1<1000 0\fi\ifnum#1<100 0\fi\ifnum#1<10 0\fi#1}%
    \StrChar{\teamnumfull}{1}[\inpA]%1
    \StrChar{\teamnumfull}{2}[\inpB]%2
    \StrChar{\teamnumfull}{3}[\inpC]%3
    \StrChar{\teamnumfull}{4}[\inpD]%4
    \StrChar{\teamnumfull}{5}[\inpE]%5
    \StrChar{\teamnumfull}{6}[\inpF]%6
    \StrChar{\problemnum{}{problem}}{1}[\inpGtmp]%7
    \StrChar{\problemnum{}{problem}}{2}[\inpHtmp]%8
    \def\inpG{\intcalcSub{\expandafter`\inpGtmp}{`A}}%
    \def\inpH{\intcalcSub{\expandafter`\inpHtmp}{`A}}%
    \FPeval{\result}{%
        round(%
            7*(\inpA+\inpD+\inpG) + %
            3*(\inpB+\inpE+\inpH) + %
            9*(\inpC+\inpF      )   %
        ,0)%
    }%
    \FPeval{\finalresult}{round(\result-trunc(\result/10,0)*10,0)}%
}

\newcommand{\barcode}{\bf\tt\LARGE%
% sazba
{\small\inpA\inpB}\inpC\inpD\inpE\inpF\problemnum{}{problem}\finalresult%
}

\newcommand{\qrcode}{%
\begin{pspicture}(1in,1in)%
%\psbarcode{https://db.fykos.cz/fyziklani116/submit/entry/000683AA3}{}{qrcode}%
\psbarcode{https://db.fykos.cz/fyziklani\eventid/submit/entry/\inpA\inpB\inpC\inpD\inpE\inpF\problemnum{}{problem}\finalresult}{}{qrcode}%
\end{pspicture}%
}

%%%%%%%%%%%%%%%%%%%%%%%%%%%%%%%%%%%%%%%%%%%
% Setting problems per page
% here set appropriate height 
%
% Parameters of geometry may be modified,
% (Note they're fine tuned to vertically center content
% (which should be done in a more generic way, however.))
%%%%%%%%%%%%%%%%%%%%%%%%%%%%%%%%%%%%%%%%%%%
\newcommand\problemtaskbox{%
\setcounter{figure}{0}
\vspace{-2mm}\par% !!MAGIC!!
\noindent\begin{minipage}[t][67mm]{\textwidth}
	\centerline{\textbf{Tým \teamid{\theteamno}} {\footnotesize (\teamname{\theteamno})}}
	\problemtask\par
\end{minipage}
%begin barcode
% 59.5mm !!MAGIC!!
% 5mm !!MAGIC!!
\barcodedef{\teamid{\theteamno}}
\vspace{-\baselineskip}\raisebox{0pt}[0pt][0pt]{\hspace{0.7\textwidth}\raisebox{59.5mm}{\hfill\barcode\hfill}}%
\par\noindent%
\vspace{-\baselineskip}\raisebox{0pt}[0pt][0pt]{\raisebox{5mm}{\qrcode}}%
\par%
%end barcode
\noindent\makebox[\linewidth]{\rule{\paperwidth}{0.5pt}}
}


% here should be correct number of problems
\begin{document}
\newcounter{teamscount}\setcounter{teamscount}{\teamscount}% pocet tymu
\newcounter{teamno}\setcounter{teamno}{0}% id teamu
\newcounter{problemno}% poradi problemu
\newcounter{problemskip}% poradi problemu
\newcounter{pagecount}\setcounter{pagecount}{0}% pocet stranek
\newcounter{pageno}% id stránky
\newcounter{pagebl}% id bloku
\newcounter{pagebllgt}% lgt bloku

% jiny celk. pocet uloh
\opt{obalka}{
  \setcounter{problemscount}{\theproblemscountenv}
}
\opt{balicek}{
  \setcounter{problem}{\theproblemscountenv}
}
% vypocet poctu stranek
\loop{
  \addtocounter{problem}{4}% 4 ulohy na str
  \stepcounter{pagecount}
}\ifnum\theproblem<\theproblemscount
\repeat

% sazba pres tymy
\loop{
  \stepcounter{teamno}
% counter init
  \setcounter{problemno}{0}
  \setcounter{problemskip}{0}
  \setcounter{pageno}{0}
  \setcounter{pagebl}{0}
  \setcounter{pagebllgt}{0}
  \opt{balicek}{
%    \setcounter{problem}{\theproblemscountenv}
    \setcounter{problemskip}{\theproblemscountenv}
  }
% neschopna rezacka (nutno vice bloku)
  \ifnum\thepagecount<\themaxcutterpaper%
    \setcounter{pagebllgt}{\thepagecount}%
  \else%
    \setcounter{pagebllgt}{\themaxcutterpaper}%
  \fi
% sazba pres stranky
  \loop{
    \newpage\noindent\makebox[\linewidth]{\rule{\paperwidth}{0.5pt}}\nopagebreak[4]
    \setcounter  {problem}{\thepageno}\addtocounter{problem}{\theproblemskip}
%    PR:\theproblem INC:\thepagebllgt
% sazba stránky
    \setcounter{problemno}{0}
    \loop{
      \ifnum\theproblem<\theproblemscount\problemtaskbox\addtocounter{problem}{-1}\fi
      \addtocounter{problem}{\thepagebllgt}
      \stepcounter{problemno}
    }\ifnum\theproblemno<4\repeat% uloh na stranku
% posledni blok musi byt kratsi
    \stepcounter{pageno}
    \ifnum\themaxcutterpaper=\thepagebllgt
      \pgfmathparse{Mod(\thepageno,\thepagebllgt)==0?1:0}\ifnum\pgfmathresult>0 
% 3*\thepagebllgt UGLY
        \addtocounter{problemskip}{\thepagebllgt}
        \addtocounter{problemskip}{\thepagebllgt}
        \addtocounter{problemskip}{\thepagebllgt}
        \pgfmathparse{(\thepagecount-\thepageno)<\thepagebllgt?1:0}\ifnum\pgfmathresult>0
          \setcounter{pagebllgt}{\thepagecount}
          \addtocounter{pagebllgt}{-\thepageno}
        \fi
      \fi
    \fi
  }\ifnum\thepageno<\thepagecount\repeat
% zbytek puvodni sazby
  \clearpage
}\ifnum\theteamno<\theteamscount\repeat
 
\end{document}
