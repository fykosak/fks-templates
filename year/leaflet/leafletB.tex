\documentclass[fykos]{fksempty}
\usepackage{bbding} 
\setcounter{year}{Y}
\setcounter{batch}{B}

\newfontfamily\dcmid[Scale=1.9]{[lmssdc10.pfb]}
\newfontfamily\dcbig[Scale=9]{[lmssbx10.pfb]}
\newcommand\mss{\sffamily\fontsize{12pt}{12pt}\selectfont}
\newcommand\mssb{\sffamily\fontsize{15pt}{15pt}\selectfont}

\newlength\logohead
\setlength\logohead{0mm}
\newcommand\nadpissptakemL[4]{\subsection[#1]{\includegraphics{logo#2.eps} \addtolength{\logohead}{#3}\raisebox{\logohead}{#1}}}
\newcommand\nadpissptakemLL[4]{\section[#1]{\includegraphics{logo#2.eps} \addtolength{\logohead}{#3}\raisebox{\logohead}{#1}}}
\makeatletter
\newcommand\seriesheading[1]{%
\section[Seriál: #1]{\centering Seriál: {#1}}}
\makeatother

\begin{document}

\noindent{\dcmid Fyzikální korespondenční seminář MFF UK \rule[4pt]{28mm}{3pt}}
\vspace{1mm}
\noindent{\includegraphics[height=20mm]{logo8.eps}\hspace{0pt plus2fil}\dcbig F\hspace{0pt plus1fil}Y\hspace{0pt plus1fil}K\hspace{0pt plus1fil}O\hspace{0pt plus1fil}S}
\vspace{5mm}
\noindent{\rule[3pt]{9cm}{3pt} \dcmid jsme tu již 26~let}
\noindent{\mss Přemýšlíte nad fyzikálními problémy, i~když jsou na první pohled obtížné?}
%\vskip5mm
%\noindent{\mss Chcete místo školy strávit dva týdny s~těmi, kdo mají stejné zájmy jako vy?}
% změna loga
\begin{wrapfigure}[0]{r}{55mm}
\includegraphics[width=55mm]{logo7.eps}
\end{wrapfigure}    
% ----------
\vskip2mm
\noindent{\mss Už se nechcete nudit ve školních lavicích?}
\vskip5mm
\noindent{\mss Experimentujete rádi?}
\vskip5mm
%\noindent{\mss Chcete vědět, kdo jsme?}
%\vskip5mm
\noindent{\mss Hledáte přátelství na celý život?}
\vskip5mm
\noindent{\mss Chcete potkat a~poznat naše zahraniční řešitele?}
\vskip5mm
\noindent{\mss Zajímají vás podrobnosti největších fyzikálních objevů?}
\vskip5mm
\noindent{\mss Zajímá vás, co se odehrává v~českých fyzikálních}
\vskip1mm
\noindent{\mss laboratořích a~jak to tam vypadá?}
%   \vskip1mm
%   \noindent{\csss ve školních lavicích?}
\vskip5mm
\noindent{\mss Chcete mít ze sebe radost?}
\vskip5mm
\noindent{\mss Chcete jet na exkurze do zahraničí?}


\vskip9mm
\noindent{\hfill\mssb Řešte naše úlohy!\hfill}
\vskip3mm

\noindent\rule{\textwidth}{1pt}

% -- konec titulní strany --
 \nadpissptakemL{Co je to {\FYKOS}?}{5}{6mm}{}


{\FYKOS} (FYzikální KOrespondenční Seminář) pro vás představuje
možnost si zajímavým způsobem rozšířit chápání fyziky a~proniknout
do dalších, dosud nepoznaných, oblastí této vědy. Seminář
organizují studenti a~zaměstnanci Matematicko-fyzikální fakulty
Univerzity Karlovy v~Praze již
%%%%%%%%%%%%%%%
šestadvacátým %
%%%%%%%%%%%%%%%
rokem. Cílem {\FYKOS}u je rozvíjet fyzikální myšlení, protože
člověk, který se umí nad (nejen fyzikálními) problémy zamyslet
a~cítí touhu se dobrat k~nějakému řešení, se uplatní všude, kde si
schopností lidského mozku cení.

% \FYKOS{} je určen všem zájemcům o~fyziku ze všech ročníků a~typů
% středních škol kdekoliv ve světě, kteří jsou schopni komunikovat
% česky, slovensky nebo anglicky.

\nadpissptakemL{Jak se stát řešitelem {\FYKOS}u?}{2}{6mm}{}

Jednoduše! Stačí se jen \textbf{zaregistrovat na našem webu a~poslat
řešení některých úloh}. Vše lze vyřídit i~klasickou poštou,
kdy nám kromě řešení pošlete i~základní kontaktní informace
(můžete využít přichystanou návratku).
Poté vám již bude zasíláno zadání dalších sérií na vámi udanou adresu.
% Můžete se samozřejmě připojit kdykoliv během roku, ale budete se muset
% smířit s~bodovým náskokem ostatních řešitelů.
% %%%
% Výběr na soustředění však probíhá podle hodnocení za pololetí,
% tudíž je nyní optimální chvíle zapojit se do semináře a~provířit
% vody stávajících řešitelů.
% %%%
% 
Není nutné posílat řešení všech úloh, i~jedna vyřešená úloha má
smysl. Řešitelé, kteří spočítají úplně všechno, jsou výjimkou.
Často je dobré poslat i~řešení, které není dotažené úplně do
konce, žádný učený z~nebe nespadl!

%\subsubsection{Proč řešit \FYKOS?}
%
%Řešením úloh \FYKOS{}u získáte praxi v~řešení fyzikálních
%problémů a~hlubší náhled na jejich podstatu. \FYKOS{} je též
%velmi vhodnou přípravou pro současné a~budoucí úspěšné řešitele
%jiných fyzikálních soutěží (Fyzikální olympiáda, SOČ apod.).
%
%Řešení {\FYKOS}u je dobrým odrazovým můstkem pro studium na
%Matematicko-fyzikální fakultě UK. Díky tomu, že poznáte
%organizátory {\FYKOS}u, si uděláte představu, jaké to je být
%studentem MFF.
%
%Pro nejlepší řešitele jsou připravena dvě soustředění, kde se
%seznámíte se spoustou nových přátel, se kterými máte jednu
%společnou zálibu~-- fyziku. Mnohá z~těchto přátelství pak
%přetrvávají během studia na VŠ i~déle.
%
%A~samozřejmě na nejlepší řešitele v~každé kategorii čekají
%hodnotné a~zajímavé ceny.
%
 %\vfil\eject
 \nadpissptakemL{Jak {\FYKOS} probíhá?}{3}{6mm}{}

Šestkrát do roka vám poštou zašleme brožurku se zadáním
tzv.~\emph{série} osmi úloh. Na jejich řešení máte zhruba pět týdnů. 
Úlohy pošlete poštou nebo nahrajete na našem webu do zadaných termínů.
My během dvou týdnů úlohy opravíme 
a~se zadáním následující série je pošleme poštou zpátky. V~den 
termínu doručení ($\approx$~termín~uploadu) se na internetu objeví autorská řešení úloh, 
proto pozdější odeslání není možné.

%\subsubsection{Jaké jsou ve {\FYKOS}u úlohy?}

%Úlohy většinou vycházejí ze znalostí středoškolských studentů
%(samozřejmě těch, co se zajímají o~fyziku) a~snaží se je
%dále rozšířit. První dvě jsou jednodušší a tvoří takzvanou 
%rozcvičku (R). Další tři bývají zpravidla obtížnější. Pátá úloha 
%nemívá tak přímočaré řešení jako první čtyři úlohy, proto jí 
%říkáme {\it problémová\/} (P). Předposlední úloha je {\it 
%experimentální\/} (E), očekává se, že zadaný experiment nejen 
%navrhnete, ale hlavně zrealizujete a~vyhodnotíte. Poslední
%úloha (S) souvisí se {\it seriálem na pokračování\/}, který vám 
%během roku přiblíží 
%%%
%komplexní čísla a jejich použití ve fyzice. Letošní seriál 
%můžete sledovat i na kanálu Fykosák na YouTube ({\tt 
%http://www.youtube.com/fykosak}).
%%%
%U každé z úloh je uvedeno její maximální bodové ohodnocení. 
%Celkem můžete za rok získat až 200 bodů.
\subsubsection{Jak mají vypadat řešení jednotlivých úloh?}

Ve správném řešení je důležité popsat a odůvodnit postup,
jímž byl získán uváděný výsledek.
Proto se nebojte psát více k~danému problému; čím více nad něčím přemýšlíme, tím lépe.

Posíláte-li řešení běžnou poštou, pište každou úlohu na
\emph{zvláštní} papír formátu~A4 (menší se nám lehce ztratí)
a~u~horního okraje jej podepište a~zřetelně označte číslo úlohy.
Je-li vaše řešení některé úlohy na více listech, očíslujte je, 
podepište a~sešijte k~sobě.

Obdobná pravidla platí i pro elektronická řešení,
ta můžete odesílat přes internetový formulář%
\footnote{\url{http://upload.fykos.cz/} -- vyžaduje přihlášení.} ve formátu PDF.
Doporučení, jak připravit hezké elektronické řešení jsou na webu.%
\footnote{\url{http://fykos.cz/ulohy/elektronicka-reseni}}



\nadpissptakemL{Jak se {\FYKOS} vyhodnocuje?}{7}{6mm}{}

Seminář je jedna velká, nepřetržitá soutěž. Za každou úlohu
obdržíte určitý počet bodů podle míry správnosti vašeho řešení
a~obtížnosti úlohy (bodová maxima najdete u~zadání úloh, 
experiment je bodován nejštědřeji). Pokud někdo vymyslí 
originální řešení či zašle skvěle propracovanou úlohu, může 
získat tzv.~\emph{bonus}, určitý počet bodů navíc
k~bodům standardním.

Řešitelé jsou rozděleni do čtyř kategorií dle ročníku,
jejž navštěvují,
přičemž studentům 1.~a~2.~ročníků se body za první dvě úlohy série (alias rozcvičku) násobí dvěma.


\subsubsection{Odměny za řešení}

Největší odměnou úspěšným řešitelům jsou soustředění (popsaná dále),
kde účastníci mj.~dostávají FYKOSí trička.

Přinejmenším třicítka nejlepších řešitelů
nebo nejlepší třetina řešitelů (podle toho, co je vyšší)
je odměněna odbornou či populární knížkou,
deskovou hrou nebo jinou hodnotnou cenou.
Tuto také dostane alespoň pět prvních řešitelů z~každé kategorie.
% Ideálním místem na předání odměn je podzimní soustředění,
% ale i kteří nepřijedou nezůstanou bez odměny a budeme se snažit dát předem na výběr z více cen, pokud to bude organizačně možné. 

Pro ty, kteří z~nejrůznějších důvodů na soustředění nepřijedou,
nebo chtějí mít triček více, máme také připravenou možnost,
jak tričko získat.
Stačí vyřešit aspoň 3~série s~celkovým ziskem minimálně 70 bodů.
Navíc za každou úlohu, za kterou dostane řešitel bonusový bod,
získá též i drobnou odměnu -- dle našich aktuálních zásob a fantazie.

Formální odměnou je \emph{Osvědčení úspěšného řešitele korespondenčního semináře},
na jehož základě může účastník požádat o~\textbf{prominutí odborné přijímací zkoušky na MFF~UK}.%
\footnote{Podrobnosti na \url{http://www.mff.cuni.cz/studium/uchazec/prijriz-bc.htm}.}
K~jeho zisku je potřeba alespoň~$"50 \%"$ celkového počtu bodů v~jednom ročníku semináře.

%\subsubsection{Publikace}

%Již se stalo tradicí, že každým rokem vydáváme ročenku, kde
%najdete pohromadě celý předcházející ročník semináře. Od zadání,
%přes řešení a~seriál na pokračování, až po výsledkovou listinu. 
%%K~dispozici jsou ročníky
%%%%%%%%%%%%%%%%%%
%XVII--XXIII.       %
%%%%%%%%%%%%%%%%%%%
%Ročenky nabízíme na akcích fakulty pro středoškoláky, jako je Den
%otevřených dveří (listopad), Jeden den s~fyzikou (únor) nebo na 
%soustředění. Ročenku od nás obdrží každý řešitel z~první poloviny 
%výsledkové listiny. 


%\nadpissptakemL{Akce {\FYKOS}u a odměny řešitelům}{6}{6mm}{}
\nadpissptakemL{Akce {\FYKOS}u}{6}{6mm}{}

\subsubsection{Soustředění}

\FYKOS{} pořádá dvakrát do roka, na jaře a~na podzim,
soustředění. Tato týdenní akce je určena pro přibližně
30~nejlepších řešitelů každého pololetí,
které vybereme ze všech kategorií,
a~probíhá v~nějakém krásném koutu naší vlasti. Účastníci
soustředění prožijí několik dní plných atraktivních přednášek
a~experimentů z~fyziky, a~aby intelektuální zátěž nebyla příliš
vysoká, odpočinou si všichni (řešitelé i~organizátoři) při hrách, 
zajímavém zážitkovém programu a~sportu.

Řešitelům, kteří v~odpovídajících třech sériích získají více
než polovinu bodů Studenta Pilného, \textbf{uhradíme část poplatku}
za soustředění a~soustředění \textbf{celé zaplatíme při zisku alespoň tří
čtvrtin bodů}.
\medskip

\subsubsection{TSAF}
TSAF (\ldots{} s~aplikovanou fyzikou)%
\footnote{\ldots = tři dny, týden atd.}
jsou sérií exkurzí po rozmanitých pracovištích spojených s~fyzikou pro šikovné řešitele.
Více o~letošním zahraničním TSAFu se dočtete v~části aktualit.

\subsubsection{Fyziklání}
FYKOSí Fyziklání%
\footnote{\url{http://fyziklani.cz/}}
je týmová soutěž pro středoškoláky.
Družstva mohou být až pětičlenná a~výjimkou nejsou ani slovenští účastníci.
Soutež se koná v~Praze na konci zimy a úkolem soutěžících
je ve třech hodinách správně vyřešit co nejvíce z~asi 40 příkladů různých
obtížnostních úrovní a oblastí fyziky. 

Pořádáme i internetovou variantu Fyziklání%
\footnote{\url{http://online.fyziklani.cz/}},
která se koná o~několik týdnů dříve a můžete takříkajíc z~tepla domova procvičit týmovou souhru.


\nadpissptakemL{{\FYKOS}í aktuality}{4}{7mm}{}\medskip

Od druhé půlky října pořádáme nejen pro řešitele sérii přednášek%
\footnote{\url{http://fykos.cz/akce/prednasky}}
s~fyzikální
tématikou~–~ve stručnosti: mechanika kladek, mechanika kapalin, statistika, kmitavý pohyb a diferenciální rovnice, aproximace.
Přednášky se budou konat v~Praze na MFF UK v~Troji každý druhý čtvrtek od 18.~října do 13.~prosince.

Rozhodli jsme se, že letošní TSAF (Týden s~aplikovanou fyzikou) pojmeme ve velkém stylu,
a~proto navštívíme zajímavá místa v zahraničí, nejspíše v~Německu a Švýcarsku.
Přesný listopadový termín a program akce naleznete na webu.%
\footnote{\url{http://tsaf.cz/}}
Co je však důležité -- \textbf{TSAFu se můžete zúčastnit i jako noví řešitelé FYKOSu},
stačí jen dobře vyřešit první sérii.

Další informace o~akcích a~aktuality kolem \FYKOS{}u se dozvíte na webu \url{http://fykos.cz/}
a~můžete nás sledovat i~na Facebooku na oficiální stránce \url{http://www.facebook.com/Fykos}.

\nadpissptakemLL{Zadání první série \Roman{year}.~ročníku}{12}{7mm}{}\medskip

Vyřešené úlohy můžete odesílat poštou do~15.~října~2012 (včetně),
elektronická řešení je možno uploadovat až do 16.~října 2012 do 20.00.
\medskip

\noindent\textbf{Nabídka}\quad
{\small
$$
    (\forall x \in \text{Řešitel})\big(\text{odeslání}(x, F) \le \text{12.\,9.} \,\wedge\,\text{úspěšnost}(x, F) = 1 - \epsilon \ztoho P[x \in \text{soustředění}] \gg 0\big)\,,
$$}
kde $F$ je první série, $P[A]$~je pravděpodobnost jevu~$A$ a $\epsilon \in \langle 0, 1/4\rangle$.

\problemtask
\problemtask
\problemtask
\problemtask
\problemtask
\problemtask
\problemtask
\problemtask
\par\noindent\emph{Poznámka}\quad Text seriálu naleznete v~závěru této brožurky.
\newpage
\input{../batch\thebatch/serial\thebatch.tex}
\vfill
\vbox to 67mm{
\noindent\raisebox{-0.3\baselineskip}{\ScissorHollowRight}\dotfill
\vfill
}
\newpage

\null
\vfill
\begin{center}
\metavar{address} 
%\includegraphics{ptak-drak.eps}
\end{center}
\vfill
%\textbf{\huge Návratka} \hrulefill\par
\addcontentsline{toc}{section}{Návratka}
\centerline{\emph{Návratka pro řešitele zasílající úlohy poštou}} %k~odstřižení
\vbox to 67mm{
\noindent\dotfill\raisebox{-0.3\baselineskip}{\ScissorHollowLeft}\par
\vfill
\centering
\noindent\begin{minipage}{0.95\textwidth}
\textbf{Řešitel}\par

\begin{tabularx}{\textwidth}{>{\vbox to 6.5mm{}}rXrX}
Jméno: & \multicolumn{3}{l}{\dotfill} \\
E-mail: & \multicolumn{3}{l}{\dotfill} \\
Datum narození: & \dotfill & Místo narození: &\dotfill\\
Doručovací adresa: & \multicolumn{3}{l}{\dotfill}\\
% & \multicolumn{3}{l}{\dotfill}\\
 & \multicolumn{3}{l}{\dotfill}
\end{tabularx}

\textbf{Škola}\par

\begin{tabularx}{\textwidth}{>{\vbox to 6.5mm{}}rXrX}
Třída: & \dotfill & Rok maturity: &\dotfill\\
Adresa: & \multicolumn{3}{l}{\dotfill}\\
% & \multicolumn{3}{l}{\dotfill}\\
% & \multicolumn{3}{l}{\dotfill}
\end{tabularx}
\end{minipage}
 

\vfill
%\vspace{-2cm}
}

% touch comment
\end{document} 

